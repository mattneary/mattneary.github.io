\documentclass{article}
\usepackage{fullpage}
\usepackage{attrib}
\usepackage{amsmath}
\usepackage{amsthm}
\usepackage{mathtools}
\DeclarePairedDelimiter{\floor}{\lfloor}{\rfloor}
\providecommand{\e}[1]{\ensuremath{\times 10^{#1}}}
\begin{document}

\begin{titlepage}
\title{On Interpretation}
\author{Matt Neary}
\maketitle
\begin{abstract}
The interpretation of languages is the key to modern Computer Science. It is through interpretation
that we are able to add layers of abstraction to computational procedures, concealing the details of
implementation in favor of more expressive forms. In the following article, a language will be 
iteratively defined and interpreted using exclusively Mathematical constructs. Through this approach,
we will remove all prerequisites from the introduction of language interpretation.
\end{abstract}
\thispagestyle{empty}
\end{titlepage}

\section{An Interpreter By Means of Algebra}
\subsection{Encoding a Language}
In order for a language to be interpreted, it needs to be rendered in a format which is easily
parsed. For example, in the case of natural language, we have arrived at a set of easily discerned
phonemes of speech, and a specific set of letters for written language. Another example was first
introduced by G\"odel in his proof of the Incompleteness Theorem; his means of encoding was a
numbering system known as G\"odel Numbering. A G\"odel Numbering system consists of a function
which assigns to each symbol of a language a Natural Number. The following is an example of his 
numbering system in action.

\begin{align*}
    &2 \; + \; 3
\\ \implies &(22, \; 01, \; 23)
\\ \implies &2^{22} \; + \; 3^{1} \; + \; 5^{23}
\\ \implies &\approx 1.19\e{16}
\end{align*}

As you can see, G\"odel made use of a prime-factorization-dependent method for his encoding. That is
to say, his encoding allowed for the original form to be recognized by means of factoring. This is a
very robust means of encoding from a theoretical stand-point, an arbitrarily large number can be a
token of the language to be encoded; however, as you have seen through our example, the encoded forms
grow extremely rapidly in magnitude.

It is for this reason that we will opt to avoid his definition in our examples, instead opting for 
a limit to be placed on the magnitude of our tokens; for now we allow that limit to be $79$. Our 
dictionary for the encoding of symbols follows.

\begin{align*}
    + &\implies 01
\\  &\dots
\\  0 &\implies 20
\\  1 &\implies 21
\\  &\dots
\\  79 &\implies 99
\end{align*}

Note that we begin with a range dedicated to symbols, of which we have only defined one, $+$. Then, we
translate each number to a number greater by $20$, as to avoid the range of symbols. A string of these
tokens will then be represented as a single number, composed of two-digit sequences corresponding to
each token.

\subsection{Evaluating an Expression}
The example we will now choose to define will consist of addition of numbers and nothing else. There
will be no sort of grouping or complex expressions, but merely the evaluation of addition represented by
a three-token encoding. 

We begin by defining some elementary functions to aid in the interpretation of our encoding. The naming 
scheme of $car$, $cdr$, $caddr$, $cddr$, and $cadr$ is derived from Lisp; for the sake of this article, 
consider $car$ to mean $head$, $cdr$ to mean $tail$, and all derived forms to be the composition of 
these primitives suggested by their lettering.

\begin{align*}
    cdr(x) &= \floor{x / 100}\footnotemark
\\  car(x) &= x - (100)cdr(x)
\\  caddr(x) &= car \cdot cdr \cdot cdr
\\  cddr(x) &= cdr \cdot cdr
\\  cadr(x) &= car \cdot cdr
\end{align*}
\footnotetext{This is known as the \emph{floor} function or \emph{greatest integer} function. It serves to reduce a fractional number to the next highest integer, e.g., $5.2 \implies 5$.}

Note that the $car$ function gets what is considered the head value, otherwise the first two digits from
the decimal point. Hence, we access them by subtracting the rounded-off higher values.

Now, we will begin by separating out the functionality of our evaluator into two componenets: reduction
of an encoded numeral, and evaluation of an addition form.

\begin{align*}
    number(x) &= x - 20
\\  eval(x) &= number(car(x)) + number(caddr(x))
\end{align*}

The above should seem pretty straightforward, assuming that you recall the meaning of $car$ and $caddr$.
Simply put, the above defition of $eval$ renders an addition form as the sum of its first token and
its third token. For example, let's encode and evaluate the form $2 + 3$.

\begin{align*}
    &eval(2 + 3)
\\  \implies &eval((22, \; 01, \; 23))
\\  \implies &eval(230122)
\\  \implies &number(car(230122)) + number(caddr(230122))
\\  \implies &number(22) + number(car(cdr(2301)))
\\  \implies &2 + number(car(23))
\\  \implies &2 + number(23)
\\  \implies &2 + 3
\\  \implies &5
\end{align*}

Of course, we achieved the desired result. However, we would now like to streamline the architecture of
our evaluator. To do this, we will consolidate the $eval$ and $number$ function into one to allow for
a more extensible function. We will be making use of a piece-wise function\footnote{Some different 
tactics were used in this translation. For example, rather than check the token type by a simple $< 20$, 
we checked whether there was a single token passed by means of $cdr(x) = 0$.}.

\begin{align*}
eval(x) = \left\{
  \begin{array}{lr}
       cdr(x) = 0, &x - 20
    \\ cadr(x) < 20, &eval(car(x)) + eval(caddr(x))
  \end{array}
\right.
\end{align*}

\subsection{A Language with Composition}
Earlier, we defined an evaluator of a language with a single expression type: binary sums. However, we
now wish to define an evaluator of nested addition. Rather than incorporate parentheses, we will make
use of Reverse-Polish Notation (RPN) in our language. The following is an example of how this structure 
translates into traditional Mathematic notation.

\begin{align*}
    &2 \; 3 \; + \; 4 \; + \; 1 \; 6 \; + \; +
\\  \implies &(((2 + 3) + 4) + (1 + 6))
\end{align*}

In order to evaluate expressions of this form, we will introduce a second parameter to the function. Our
addition will allow for the evaluator to manage a stack of values which it will render with addition.
However, first we will define a function\footnote{This may be your first time seeing a function of multiple 
variables. Note that the mechanics remain the same, the primary difference being its rendering as a graph.} 
for the creation of token-chains called $cons$.

$$cons(x, y) = x \; + \; y(100)$$

Note the relationships between $cons$ and the $car$ and $cdr$ functions.

\begin{align*}
    car(cons(x, y)) &= x
\\  cadr(cons(x, y)) &= y
\\  cdr(cons(x, cons(y, z))) &= cons(y, z)
\\  cons(car(cons(x, y)), cdr(cons(x, y))) &= cons(x, y)
\end{align*}

Now we will define our new $eval$ function accepting two arguments. The first argument remains the 
expression, but the second is a stack of values.

\begin{align*}
eval(x, s) = \left\{
  \begin{array}{lr}
        x = 0, &car(s)
    \\  car(x) < 20, &eval(cdr(x), cons(car(s) + cadr(s), \; cddr(s)))
    \\  car(x) \ge 20, &eval(cdr(x), cons(car(x) - 20, \; s))
  \end{array}
\right.
\end{align*}

Notice that once the entire expression has been read, the first value in the stack is returned. 
Otherwise, if the character is the addition operator, the rest of the expression is evaluated with the
first two values on the stack added. Lastly, if a numeric value is read, the function recurses upon the
rest of the expression with the number pushed to the stack.

\subsection{A Language with Multiple Forms}
We have already expanded out language to allow for arbitrary nesting of expressions. Now, we will set
out to add the additional operators of Mathematics. First, we expand our dictionary of encoding.

\begin{align*}
    + &\implies 01
\\  \cdot &\implies 02
\\  - &\implies 03
\\  / &\implies 04
\end{align*}

Now, we expand our piece-wise evaluator to account for these operators. The evaluator will be based
on an $op$ function which we define as follows.

\begin{align*}
op(o, a, b) = \left\{
  \begin{array}{lr}
        o = 1, &a + b
    \\  o = 2, &a \cdot b
    \\  o = 3, &a - b
    \\  o = 4, &a / b
  \end{array}
\right.
\end{align*}

Now we incorporate $op$ into our evaluator.

\begin{align*}
eval(x, s) = \left\{
  \begin{array}{lr}
        x = 0, &car(s)
    \\  car(x) < 20, &eval(cdr(x), cons(op(car(x), cadr(s), car(s)), \; cddr(s)))
    \\  car(x) \ge 20, &eval(cdr(x), cons(car(x) - 20, \; s))
  \end{array}
\right.
\end{align*}

With the evaluator defined, we will walk through another example.

\begin{align*}
    &2 \; 3 \; \cdot \; 2 \; + \; 2 \; -
\\  \implies &eval((22, 23, 02, 22, 01, 22, 03), 0)    
\\  \implies &eval(03220122022322, 0)
\\  \implies &eval(032201220223, 02)
\\  \implies &eval(0322012202, 0203)
\\  \implies &eval(03220122, 06)
\\  \implies &eval(032201, 0602)
\\  \implies &eval(0322, 08)
\\  \implies &eval(03, 0802)
\\  \implies &eval(0, 06)
\\  \implies &6
\end{align*}

We arrived at the correct result, and hopefully the methodologies of the $eval$
function were made clearer. Finally, we will now prove that the interpreter
does its job: That it accurately evaluates all expressions of the language's
grammar.

\subsection{Verification}

Our first step is to define a variation on $eval$, called $eval'$.  This
altered form will instead bear the full stack when finished evaluating.

\begin{align*}
eval'(x, s) = \left\{
  \begin{array}{lr}
        x = 0, &s
    \\  car(x) < 20, &eval'(cdr(x), cons(op(car(x), cadr(s), car(s)), \; cddr(s)))
    \\  car(x) \ge 20, &eval'(cdr(x), cons(car(x) - 20, \; s))
  \end{array}
\right.
\end{align*}

We will begin with a related proof. We wish to show that for all expressions
$e$ in the grammar, $eval'$ bears a stack with the evaluation of $e$ at the
top.

\begin{proof}

To begin, we wish to show that for all stacks $s$, $n \in [0, 79]$ and $e$ such
that $e$ is the expression representation of $n$, $eval'(e, s)$ bears a stack
with $n$ at the top. We know that $e = n + 20$, and thus have that $eval'(e, s)
= eval'(n + 20, s)$. Now, since $n + 20 \ge 20$, we have that $eval'(e, s) =
eval'(0, cons(n + 20 - 20, s)) = cons(n, s)$.

Now, given two expressions $e_1$ and $e_2$ such that their evaluations bear a
stack with their values at the top, we wish to show that for all stacks$s$,
$eval'(cons(e_1, eval'(e_2, o)), s) = op(o, car(eval'(e_1, s)), car(eval'(e_2,
s)))$. This follows trivially from the inductive hypothesis and definition of $eval'$.

We now conclude that for all expressions $e$ in the language and stacks $s$,
$eval'(e, s)$ bears a stack with the value of $e$ at the top.

\end{proof}

Using the above proof, we declare that for all expressions $e$ in the language
and stacks $s$, $eval(e, s)$ bears the value of $e$. 

\end{document}
