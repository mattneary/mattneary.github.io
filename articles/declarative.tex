\documentclass{article}
\usepackage{attrib}
\usepackage{amsmath}
\usepackage{mathtools}
\begin{document}

\begin{titlepage}
\title{On the Declarative Paradigm}
\author{Matt Neary}
\maketitle
\begin{abstract}

\end{abstract}
\thispagestyle{empty}
\end{titlepage}

\section{Introduction to Declarative Programs}
In the design of formal programs, there are two general approaches. Some programs
are considered \emph{imperative}, and others \emph{declarative}. The distinction
between the two is the subject matter of this article. 

You will often hear the declarative paradigm described as concerned with
``what, not how.'' This is usually interpreted to mean that declarative programs
are unconcerned with the details of implementation. This concept, however, can
be better related as a definition of relationships rather than a means of
construction. That may not seem to clarify the matter, but throughout this
article we will aim to display this aspect of declarative programming quite
clearly.

Our language of choice is Haskell, the epitome of functional, declarative 
programming and, in general, a very powerful and quality language. We start with
a trivial example of how this aspect of declarative programming takes form.

\begin{verbatim}
fact 0 = 1
fact n = n * (fact (n - 1))
\end{verbatim}

Of course, the above is a definition of a factorial function. You will notice
its striking similarity to an inductive defintion in Mathematics. This similarity
is due to the fact that an inductive defintion is in itself a very declarative
idea. An inductive definition describes an initial case and the relationship of
subsequent cases in terms of their predecessors.

With this idea of relationships over construction in mind, we will now dive
into the viability of this approach in a building of complex structures and
as an approach to programming.

\section{Axioms and Building}

Any assurance of consistency in our Mathematical systems is dependent upon the
Axiomatic Method. This method assures that assumptions are minimized, yet given a 
few means of extrapolation from one idea to another, we remain able to achive
complex truths.

The whole of Mathematics has been given rigorous foundation in this idea, through
many efforts. We will focus on the Zermelo–Fraenkel set theory with the Axiom of 
Choice (ZFC), a specific foundation of Mathematics. ZFC begins with a series of
Axioms from which all of Mathematics can be derived. For example, the Axiom of
Pair states that for all values $A$ and $B$, there exists a set such $P$ that a 
value is in $P$ if and only if it is either equal to $A$ or equal to $B$.

As you can tell, the Axiom of Pair is a rule for construction. Hence, by our
earlier distinction between imperative and declarative programs, we can conclude
that is an imperative construct.

\section{Equivalency}

\end{document}
