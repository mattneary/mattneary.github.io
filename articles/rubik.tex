\documentclass{article}
\usepackage{attrib}
\usepackage{amsmath}
\usepackage{mathtools}
\DeclarePairedDelimiter{\floor}{\lfloor}{\rfloor}
\providecommand{\e}[1]{\ensuremath{\times 10^{#1}}}
\begin{document}

\begin{titlepage}
\title{Deriving Macros for the Solving of a Rubik's Cube}
\author{Matt Neary}
\maketitle
\begin{abstract}
Solving a Rubik's cube is undoubtedly non-trivial; however, with the right 
motivations, one can find algorithms that will manipulate the cube in useful 
ways, allowing one to gradually approach a solved cube. Using basic group 
theory we will analyze the permutation of facelets resultant of a face turn, 
and then the products of various permutation compositions. Having recognized 
the rearrangement of facelets, we can identify the area of manipulation and 
work toward matching our desired region of change to that area.
\end{abstract}
\end{titlepage}

\section{Notation}
In our discussion of the Rubik's cube we will need a means of referring both 
(1) to the \emph{facelets} by which the cube is composed, and (2) to the 
manipulations of the cube which are possible.

\subsection{The State of a Cube}
A 2x2 Rubik's cube consists of 6 faces, each with two layers of 2 cubes each. 
To identify a given facelet, i.e., a colored square on one of the six faces, 
we provide the following notation.

\begin{align*}
	<face> & ::= \ F\ |\ U\ |\ D\ |\ L\ |\ R\ |\ B
\\	<edge> & ::= \ U\ |\ D
\\	<corner> & ::= \ L\ |\ R
\\
\\	<facelet> & ::= \ <face> \; <edge> \; <corner>
\end{align*}

This notation is sufficient to fully identify the state of a cube.

\subsection{Manipulations of a Cube}
Beyond identifying the state of a cube, we will further want a means of 
communicating manipulations of the cube, i.e., face-turns. There are 6 face-
turn operations, one for each face of the cube. We will articulate each as a 
permutation of facelets.

\begin{align*}
	U & := (\text{LUL FUL RUL BUL})(\text{LUR FUR RUR BUR})(\text{UUL UUR UDR UDL})
\\	D & := (\text{LDL FDL RDL BDL})(\text{LDR FDR RDR BDR})(\text{DUL DUR DDR DDL})
\\	L & := (\text{UUL FUL DUL BDR})(\text{UDL FDL DDL BUR})(\text{LUL LUR LDR LDL})
\\	R & := (\text{UUR FUR DUR BDL})(\text{UDR FDR DDR BUL})(\text{RUL RUR RDR RDL})
\\	F & := (\text{UDL RDL DUL LUR})(\text{UDR RUL DUR LDR})(\text{FUL FUR FDR FDL})
\\	B & := (\text{UUL RUR DDL LDL})(\text{UUR RDR DDR LUR})(\text{BUL BUR BDR BDL})
\end{align*}

Face-turns are the atomic manipulations of a cube. Howevever, much more 
complex manipulations are possible. We will later provide an inductive 
definition of the set of all manipulations, or permutations.

\section{Groups}
\subsection{Definition of a Group}
A group $\mathcal{G}$ consists of an underlying set and a binary operator 
$\*$ on members of that set meeting the following criteria.

\begin{enumerate}
  \item The operation $\*$ is closed on $\mathcal{G}$, so for $h,g \in \mathcal{G}$, we have $h * g \in \mathcal{G}$.
  \item The operation $\*$ is asociatve, so for $h,g,f \in \mathcal{G}$, we have $(f * g) * h = f * (g * h)$.
  \item There exists an identity element $\mathcal{I} \in \mathcal{G}$ such that $(\forall g \in \mathcal{G})(\mathcal{I} * g = g * \mathcal{I} = g)$.
  \item Every element $g \in \mathcal{G}$ has an inverse relative to $\*$, $g^{-1}$ such that $g * g^{-1} = g^{-1} * g = \mathcal{I}$.
\end{enumerate}

\subsection{Theorems of Groups}
\begin{enumerate}
  \item The identity element $\mathcal{I}$ is unique.
  \item If $A * B = \mathcal{I}$, $A = B^{-1}$.
  \item If $A * X = B * X$, $A = B$.
  \item The inverse of $A * B$ is $B^{-1} * A^{-1}$,
  \item $(A^{-1})^{-1} = \mathcal{I}$,
\end{enumerate}

\section{The Group of Cube Permutations}
Obviously face-turns of a Rubik's cube are not commutative; if they were, any 
cube could be solved with no more than 2 turns of each face! This is the 
hallmark of groups and thus we have reason to believe that Rubik's cube 
manipulation would be well modeled by a group. We will begin by defining all 
possible cube manipulations, i.e., sequences of face-turns.

\begin{align*}
	< \text{faceturn}>  & ::= U\ |\ D\ |\ L\ |\ R\ |\ F\ |\ B
\\	< \text{permutation}>  & ::= < \text{faceturn}>  | < \text{permutation}> < \text{faceturn}> 
\end{align*}

The set of all of these \emph{permutations} translates well into a group 
which we will call $\mathcal{R}$, so long as we consider (1) $AB$ to be the 
composition of permutations $A$ and $B$, and (2) equality to be 
\emph{extensive}, i.e., two permutations that bear the same manipulation to 
be equal. Now obviously this group represents all possible states of a cube, 
as face-turns and their composition are the only manipulations possible to 
perform on a cube, hold removing the stickers.

\subsection{The Binary Operation on Cube Permutations}
In our group $\mathcal{R}$, the binary operation $\*$ will be the composition 
of two permutations. The operation is obviously closed on $\mathcal{R}$ given 
our prior inductive definition of permutations. Furthermore, $\*$ is 
associative as there is no concept of grouping of Rubik's cube face-turns. 
The identity, $\mathcal{I}$, is simply the identity permutation, e.g., 
$RRRR$; recall our *extensive* interpretation of equality, its inclusion 
ensures that the identity is unique.

\subsection{Inverses}
All permutations $A$ in the group $\mathcal{R}$ have an inverse denoted as 
$A^{-1}$. Recall from the theorems of groups that the inverse of a product 
over a group $A * B$ is equal to $B^{-1} * A^{-1}$. If you merely imagine 
performing two consecutive moves on a Rubik's cube, the reversed order of 
their inverse should be intuitive.

The inverse of a face-turn is that face-turn applied three times, or 
equivalently, in the opposite direction. This is trivial to show by working 
out the permutation compositions or simply by referencing the nature of 
cyclical permutations. For example, 
$\mathcal{I} = (\text{A B C D})^4 = (\text{A B C D})^3 * (\text{A B C D})$, 
and thus by Theorem 2, $(\text{A B C D})^{-1} = (\text{A B C D})^3$.

\section{Subgroups of Cube Permutations}
In our definition of $\mathcal{R}$, we defined a permutation as a composition 
of face-turns. Were we to restrict the face-turns of which they were 
composed, we would have made a \emph{subgroup} of $\mathcal{R}$. The 
following will serve as an example.

\begin{align*}
	< \text{faceturn}>  & ::= F^2R^2
\\	< \text{permutation}>  &::= < \text{faceturn}>  | < \text{permutation}> < \text{faceturn}> 
\end{align*}

In this case we would call the set ${F^2R^2}$ the \emph{generator} of the 
subgroup. This subgroup happens to have 6 elements, as you will find by 
repeating the move $F^2R^2$ until returning to the identity. The size of the 
generated subgroup always coincides with the cycle of the generator, i.e., 
the \emph{order} of the permutation.


\section{Permutations}
We have already discussed the permutation of facelets caused by each face-
turn. Now, having inductively defined the set of all permutations, we will 
discuss them on a more complex level.

\subsection{Parity}
We wish to define \emph{parity} of a permutation as the number of 2-cycle 
permutations of which the permutation is composed. This can in turn be 
interpreted as the number of swapped facelets. We define a function for this 
purpose, \emph{par}, as follows.

\begin{align*}
	par &(\text{A B}) = 1
\\	par &(\text{A } \cdots \text{ B}) = (par\ (\text{A } \cdots)) + 1
\end{align*}

Given this inductive definition of parity, we can show that if a permutation 
has $n$ elements, it will have parity $n-1$. If the parity is odd, a 
permutation is said to be odd, and if even, even.

The parity of each face-turn is the sum of the parity of the two constituent 
permutations, i.e., $(4-1)+(4-1) = 6$. Thus all face-turns are even 
permutations. Now since a permutation on the cube is defined inductively on 
face-turns, all permutations are even. Thus we will say each state of the 
cube has even parity.

We have shown all permutations on the cube to be of even parity. This means 
that no two facelets can be swapped without side-effects which will be of 
guidance in our attempts to solve the cube. We thus should expect cycles of 
length three, or \emph{3-cycles}, to be significant in our solving of the 
cube.

\section{Algorithms}
In order to solve a Rubik's cube, our goal will be to find move sequences 
that perform a simple manipulation of minimal side-effects which we will call 
\emph{algorithms}. The \emph{support} of an algorithm will be the set of 
facelets or cubies permuted by its moves.

\subsection{Methodology}
The first layer of a 2x2 Rubik's cube can be solved quite easily with 
intuition. However, having solved the first layer it becomes clear that 
maintaining your progress as you attempt to finish the cube will not be an 
easy task. Ignoring the details of implementation, it should be clear that we 
aim to fulfill the following tasks.

\begin{enumerate}
  \item Solve the First Layer
  \item Orient the Bottom Layer
  \item Permute the Bottom Layer
\end{enumerate}  

\subsection{Solving the First Layer}
In solving the first layer, the goal is to find permutations that move a 
bottom layer cubie in any orientation up to the top layer correctly oriented. 
Given the permutations brought about by the face-turns, and the 
insignificance of side-effects, this should not be too difficult. The key to 
keep in mind is that the support of a side turn excludes cubies on the 
opposite side. That may seem obvious, but may be looked over as a means of 
$caching$ a top-colored cubie as you bring down the top layer, then allowing 
you to place the cubie and reset the side face.

\subsection{Permuting the Bottom Layer}
After the First Layer has been solved, you are probably struck by the variety 
of colors on the bottom face. We, however, will first address the position of 
cubies, rather than whether they are pointing in the right direction.

Hence our goal is to permute the bottom cubies with a support independent of 
the first layer. The most intuitive route to swapping corners would be to 
bring a back corner toward the front of last layer, \emph{cache} it by 
rotating the last layer, and to then return the rotated side and top faces. 
This would be achieved by the following algorithm.

$$L U^{-1} L^{-1} U$$

Analysis of this move sequence shows the permutation to be the following.

$$(\text{LUL LUR BUL UDL UUL FUL})\ (\text{DDR BUR LDL UUR BDL RUR})$$

If we reduce the above to refer to the permutation of *cubies* we have a simpler view of the action performed, as follows.

$$(\text{UDL UUL})\ (\text{UUR DDR})$$

The $UDL$ and $UUL$ cubies were swapped as intended; however, there was the 
unintended side-effect swapping $UUR$ with $DDR$. This side-effect extended 
the support of our move sequence to the first layer. Referencing the 
permutation listed above, we see that the side-effect originates in the $UUR$ 
cubie.

To counter-act this side-effect we will look at the moves directing the cubie 
to the first layer. The second half of the move sequence, $L^{-1} U$ performs 
the following permutation, in terms of cubies.

$$(\text{UDR UUR UUL})\ (\text{DDR UDL DUR})$$

Thus prior to our resetting of the cube we should prepare the support that the $DUR$ cubie will eventually end up in the $DUR$ slot. $DUR$ should go, therefore, to the $UDL$ slot. The move sequence for this manipulation is quite obvious, and as follows.

$$R^{-1} U$$

Let's look at the permutation of our new move sequence, including the 
preparatory steps in the middle.

$$L U^{-1} R^{-1} U L^{-1} U = (\text{UUL DUL DDL UDL})\ (\text{UUR UDR})$$

Notice that once again we have undesirable side-effects, this time caused by 
our intermediate step. Regarding cubies, the issue can be resolved by 
bringing $UDL$ to the $DDL$ slot and $DUL$ up to the last layer. This means 
aligning the last layer left above the first layer left, rotating the left 
side so that the two up edge cubies are on the bottom, and resetting the last 
layer. Thus we have the following final sequence to our algorithm.

$$U^2 R U^2$$

Hence we have the following final algorithm, after combining $U$ and $U^2$ 
into $U^{-1}$.

$$L U^{-1} R^{-1} U L^{-1} U^{-1} R U^2$$

Our final permutation is the following; beyond swapping the right cubies, it 
merely re-orients the left cubies. Recall that a swap without side-effects 
would be impossible, as discussed in our digression into parity.

$$(\text{UUL LUL BUL})\ (\text{RUL RUR FUR UUR UDR BUR})\ (\text{FUL LUR UDL})$$

\subsection{Orienting the Bottom Layer}
With corners in their correct positions, our goal is now to correctly orient 
them while maintaining position. If the goal is to change orientation of the 
bottom pieces, the natural approach would be to rotate a side toward the 
bottom, \emph{cache} the now rotated cubie, and then move back the side and 
bottom faces. Note that we will write all algorithms in the context of 
bottom-face facing up, i.e., $U$ refers to a turn of the last layer. Our 
initial idea of an orientation algorithm would then translate into the 
following move sequence. 

$$R U^2 R^{-1} U^2$$

Note the need for the last layer to be turned twice to perform our desired 
caching. The performed permutation can be found to be as follows. 

$$(\text{FUL UUL FDR UDR BUR})\ (\text{DUL FUR UUR UDL BUL})\ (\text{RDL RUL RUR LUR LUL})$$

This may not be easily parsed. A summary of this permutation in terms of 
*cubies* is as follows.

$$(\text{UUR UDL UUL DUL UDR})$$

Thus we have a \emph{support} of $\{ \text{UUR, UDL, UUL, DUL, UDR} \}$. This 
translates into an unwanted side-effect. To combat this we will aim to 
prepare the cube in a way such that this side-effect undoes a prior 
manipulation, essentially preemptively inverting the side-effect.

The most obvious means to prepare the support is to place in the $UUL$ slot 
the $DUL$ cubie; to do this we should simply bring $DUL$ up to the last 
layer, \emph{cache} it to the side, move back the side we brought up, then 
shift it from $UDL$ to $UUL$. This translates into the following move 
sequence.

$$R U R^{-1} U$$

Putting together our support preparation with our last layer orientation we 
have a useful algorithm.

$$R U R^{-1} U R U^2 R^{-1} U^2$$

For the sake of completeness, here is the product of our two permutations.

$$(\text{BUL UUL LUL})\ (\text{UUR BUR RUR})\ (\text{FUR UDR RUL})$$

As you can see, our algorithm now serves to change the orientation of $UUL$, 
$UUR$, and $UDR$.

\section{Conclusion}
We have successfully used basic group theory to design move patterns which 
will aid in solving a Rubik's cube. Most often these sequences of moves would 
be memorized, but you have seen that they can be just as well derived with a 
little logic, strategy, and permutation analysis.

\end{document}
